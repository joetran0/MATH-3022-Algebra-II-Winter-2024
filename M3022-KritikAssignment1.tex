\documentclass[11pt]{amsart}
\usepackage[utf8]{inputenc}
\usepackage{amsfonts, amssymb, amsmath, amsthm, booktabs, hyperref, pgfplots, tikz, xcolor, mathrsfs, nicefrac, multicol}
\usepackage[margin=1in]{geometry}

\theoremstyle{}\newtheorem{question}{Question}
\theoremstyle{definition}\newtheorem*{solution}{Solution}

\setlength{\parindent}{0pt}
\setlength{\parskip}{2.5pt}

\newcommand{\K}{\mathbb{K}}
\newcommand{\C}{\mathbb{C}}
\newcommand{\R}{\mathbb{R}}
\newcommand{\Q}{\mathbb{Q}}
\newcommand{\Z}{\mathbb{Z}}
\newcommand{\N}{\mathbb{N}}
\newcommand{\A}{\mathcal{A}}
\newcommand{\B}{\mathcal{B}}
\newcommand{\F}{\mathcal{F}}
\newcommand{\G}{\mathcal{G}}
\renewcommand{\L}{\mathscr{L}}
\newcommand{\M}{\mathscr{M}}
\renewcommand{\P}{\mathcal{P}}
\newcommand{\T}{\mathcal{T}}
\newcommand{\U}{\mathcal{U}}
\newcommand{\V}{\mathcal{V}}
\newcommand{\W}{\mathcal{W}}
\newcommand{\e}{\varepsilon}
\renewcommand{\O}{\mathcal{O}}
\renewcommand{\SS}{\mathcal{S}}
\renewcommand{\phi}{\varphi}

\DeclareMathOperator{\diam}{diam}
\DeclareMathOperator{\dist}{dist}
\DeclareMathOperator{\cluster}{cluster}

\begin{document}

\noindent \textbf{MATH 3022 Algebra II} \hfill \textbf{Kritik Assignment 1} \\
\noindent \textsc{Assignment} \hfill \textsc{Joe Tran}

\vspace{5pt}

\begin{question}
    What is the fundamental difference between groups and rings?
\end{question}

\begin{solution}
    In Group Theory, we study a set $G$ with a \emph{single} binary operation $* : G \times G \to G$, in which we call the pair $(G, *)$ a group. Here in Ring Theory, we study a set $R$ with \emph{two} binary operations, namely $+$ and $\cdot$, in which we call $R$ a ring.
\end{solution}

\begin{question}
    Give two characterizations of an integral domain.
\end{question}

\begin{solution}
    Two characterizations of an integral domain is as follows:
    \begin{enumerate}
        \item The Cancellation Law: If $D$ is a commutative ring with identity, then $D$ is an integral domain if and only if for every nonzero $d \in D$ with $da = db$, then $a = b$.
        \item $R$ is an integral domain if for every $x, y \in R$ such that $xy = 0$, then either $x = 0$ or $y = 0$.
    \end{enumerate}
\end{solution}

\begin{question}
    Provide two examples of fields, one infinite, one finite.
\end{question}

\begin{solution}
    We have the two examples of fields below:
    \begin{enumerate}
        \item For the example where a field is finite, take $F$ to be the set of matrices
        \begin{equation*}
            F = \left\{\begin{bmatrix} 1 & 0 \\ 0 & 1 \end{bmatrix}, \begin{bmatrix} 1 & 1 \\ 1 & 0 \end{bmatrix}, \begin{bmatrix} 0 & 1 \\ 1 & 1 \end{bmatrix}, \begin{bmatrix} 0 & 0 \\ 0 & 0 \end{bmatrix}\right\}
        \end{equation*}
        This is a (finite) field with entries in $\Z_2$.

        \item The set $\Q[\sqrt{2}] = \{a + b\sqrt{2} : a, b \in \Q\}$ is a (infinite) field.
    \end{enumerate}
\end{solution}

\end{document}