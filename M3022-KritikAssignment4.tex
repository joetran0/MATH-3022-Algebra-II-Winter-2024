\documentclass[11pt]{amsart}
\usepackage[utf8]{inputenc}
\usepackage{amsfonts, amssymb, amsmath, amsthm}

\newtheorem{question}{Question}

\title{MATH 3022 Kritik Assignment 3}
\date{\today}

\begin{document}

\maketitle

\begin{question}
    Why do the axioms of a vector space appear to only have four conditions, rather than the ten you may have seen the first time you saw an axiomatic definition?
\end{question}

Vector spaces have a similar structure compared to groups and rings, in which we are able to use the axioms defined for groups and rings and apply it for vector spaces.

\begin{question}
    The set $V = \mathbb{Q}(\sqrt{2}) = \{a + b\sqrt{11} : a, b \in \mathbb{Q}\}$ is a vector space. Carefully define the operations on this set that will make this possible. Describe the subspace spanned by $S = \{\vec{u}\}$, where $\vec{u} = 3 + \frac{2}{7}\sqrt{11} \in V$.
\end{question}

Let $\vec{u}, \vec{v} \in V$. Then we have the following operations defined by the mappings: $+ : V \times V \to V$ and $\cdot : V \times V \to V$ where,
\begin{equation*}
    \vec{u} + \vec{v} = (a + b\sqrt{11}) + (c + d\sqrt{11}) = (a + c) + (b + d)\sqrt{2}
\end{equation*}
and
\begin{equation*}
    \vec{u} \cdot \vec{v} = (a + b\sqrt{11})(c + d\sqrt{11}) = (ac + 11bd) + (ad + bc)\sqrt{11}
\end{equation*}
where $a + c$, $b + d$, $ac + 11bd$, and $ad + bc \in \mathbb{Q}$.

The subspace spanned by $S = \{3 + \frac{2}{7}\sqrt{11}\}$ is the set of the form
\begin{equation*}
    W = \left\{3a + \frac{2}{7}b\sqrt{11} : a, b \in \mathbb{Q}\right\}
\end{equation*}
for some coefficients $a, b \in \mathbb{Q}$.


\end{document}