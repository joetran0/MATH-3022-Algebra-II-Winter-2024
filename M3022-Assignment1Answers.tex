\documentclass[11pt]{amsart}
\usepackage[utf8]{inputenc}
\usepackage[margin=1in]{geometry}
\usepackage{amsfonts, amssymb, amsmath, amsthm, booktabs, hyperref, pgfplots, tikz, xcolor, multicol}

\theoremstyle{definition}\newtheorem{question}{Question}
\theoremstyle{definition}\newtheorem{claim}{Claim}
\theoremstyle{remark}\newtheorem*{solution}{Solution}

\newcommand{\K}{\mathbb{K}}
\newcommand{\C}{\mathbb{C}}
\newcommand{\CC}{\mathcal{C}}
\newcommand{\R}{\mathbb{R}}
\newcommand{\Q}{\mathbb{Q}}
\newcommand{\Z}{\mathbb{Z}}
\newcommand{\N}{\mathbb{N}}
\newcommand{\F}{\mathbb{F}}
\renewcommand{\SS}{\mathcal{S}}
\newcommand{\T}{\mathcal{T}}
\newcommand{\I}{\mathcal{I}}
\newcommand{\M}{\mathcal{M}}
\newcommand{\teq}{\trianglelefteq}
\DeclareMathOperator{\diag}{diag}

\everymath{\displaystyle}

\begin{document}

\noindent \textbf{MATH 3022 Algebra II} \hfill \textbf{Homework Assignment 1} \\
\textsc{Assignment} \hfill \textsc{Joe Tran}

\begin{question}\label{question:1}
    We define two operations $\boxplus$ and $\boxtimes$ on $\Z$ as
    \begin{align*}
        a \boxplus b &= a + b - 1 \\
        a \boxtimes b &= ab - a - b + 2
    \end{align*}
    for $a, b \in \Z$.
    \begin{itemize}
        \item[(a)] Show that $\Z$ together with addition $\boxplus$ and multiplication $\boxtimes$ is a ring.
        \item[(b)] Determine if this ring is
        \begin{itemize}
            \item[(i)] A commutative ring
            \item[(ii)] A ring with identity
            \item[(iii)] An integral domain.
            \item[(iv)] A field.
        \end{itemize}
    \end{itemize}
\end{question}

\begin{solution}
    (a) To see that $(\Z, \boxplus, \boxtimes)$ is a ring, we verify the six properties of a ring.
    \begin{itemize}
        \item[(i)] (Associativity Over $\boxplus$) Let $x, y, z \in \Z$ be arbitrary. We show that
        \begin{equation*}
            (x \boxplus y) \boxplus z = x \boxplus (y \boxplus z)
        \end{equation*}
        Using the properties of the usual addition on $\Z$,
        \begin{align*}
            (x \boxplus y) \boxplus z &= (x + y - 1) \boxplus z \\
            &= (x + y - 1) + z - 1 \\
            &= x + y - 1 + z - 1 \\
            &= x + y + z - 1 - 1 \\
            &= x + (y + z - 1) - 1 \\
            &= x + (y \boxplus z) - 1 \\
            &= x \boxplus (y \boxplus z)
        \end{align*}
        \item[(ii)] (Identity Over $\boxplus$) We claim that the identity over $\boxplus$ is $1 \in \Z$ because for any $x \in \Z$,
        \begin{equation*}
            x \boxplus 1 = x + 1 - 1 = x
        \end{equation*}
        and
        \begin{equation*}
            1 \boxplus x = 1 + x - 1 = x
        \end{equation*}
        \item[(iii)] (Inverse Over $\boxplus$) We first find the inverse element over $\boxplus$. Let $x \in \Z$ be arbitrary. Then there exists an element $y \in \Z$ such that $x \boxplus y = 1$. Then
        \begin{align*}
            x \boxplus y &= 1 \\
            x + y - 1 &= 1 \\
            y &= 2 - x
        \end{align*}
        So the inverse element over $\boxplus$ is $2 - x \in \Z$. Furthermore,
        \begin{equation*}
            x \boxplus (2 - x) = x + (2 - x) - 1 = x + 2 - x - 1 = 1
        \end{equation*}
        and
        \begin{equation*}
            (2 - x) \boxplus x = (2 - x) + x - 1 = 2 - x + x - 1 = 1
        \end{equation*}
        \item[(iv)] (Abelian Over $\boxplus$) Let $x, y \in \Z$ be arbitrary. We show that
        \begin{equation*}
            x \boxplus y = y \boxplus x
        \end{equation*}
        Indeed,
        \begin{equation*}
            x \boxplus y = x + y - 1 = y + x - 1 = y \boxplus x
        \end{equation*}
        \item[(v)] (Associativity Over $\boxtimes$) Let $x, y, z \in \Z$ be arbitrary. Then we show that
        \begin{equation*}
            (x \boxtimes y) \boxtimes z = x \boxtimes (y \boxtimes z)
        \end{equation*}
        Indeed,
        \begin{align*}
            (x \boxtimes y) \boxtimes z &= (xy - x - y + 2) \boxtimes z \\
            &= (xy - x - y + 2)z - (xy - x - y + 2) - z + 2 \\
            &= xyz - xz - yz + 2z - xy + x + y - 2 - z + 2 \\
            &= xyz - xy - xz + x - yz + y + z - 2 + 2 \\
            &= xyz - xy - xz + 2x - x - yz + y + z - 2 + 2 \\
            &= x(yz - y - z + 2) - x - (yz - y - z + 2) + 2 \\
            &= x \boxtimes (yz - y - z + 2) \\
            &= x \boxtimes (y \boxtimes z)
        \end{align*}
        as required.
        \item[(vi)] (Distributive Property) Let $x, y, z \in \Z$ be arbitrary. Then we show that
        \begin{equation*}
            x \boxtimes (y \boxplus z) = (x \boxtimes y) \boxplus (x \boxtimes z)
        \end{equation*}
        and
        \begin{equation*}
            (x \boxplus y) \boxtimes z = (x \boxtimes z) \boxplus (y \boxtimes z)
        \end{equation*}
        Indeed,
        \begin{align*}
            x \boxtimes (y \boxplus z) &= x \boxtimes (y + z - 1) \\
            &= x(y + z - 1) - x (y + z - 1) + 2 \\
            &= xy + xz - x - x - y - z + 1 + 2 \\
            &= xy + xz - x - x - y - z + 2 - 1 + 2 \\
            &= (xy - x - y + 2) + (xz - x - z + 2) - 1 \\
            &= (x \boxtimes y) + (x \boxtimes z) - 1 \\
            &= (x \boxtimes y) \boxplus (x \boxtimes z)
        \end{align*}
        and similarly,
        \begin{align*}
            (x \boxtimes y) \boxtimes z &= (x + y - 1) \boxtimes z \\
            &= (x + y - 1)z - (x + y - 1) - z + 2 \\
            &= xz + yz - z - x - y + 1 - z + 2 \\
            &= xz + yz - z - x - y + 2 - 1 - z + 2 \\
            &= (xz - x - z + 2) + (yz - y - z + 2) - 1 \\
            &= (x \boxtimes z) + (y \boxtimes z) - 1 \\
            &= (x \boxtimes z) \boxplus (y \boxtimes z)
        \end{align*}
    \end{itemize}
    Therefore, we have show that $(\Z, \boxplus, \boxtimes)$ is a ring.

    (b) (i) We claim that $(\Z, \boxplus, \boxtimes)$ is a commutative ring. Let $x, y \in \Z$ be arbitrary. Then we show that
    \begin{equation*}
        x \boxtimes y = y \boxtimes x
    \end{equation*}
    Indeed,
    \begin{equation*}
        x \boxtimes y = xy - x - y + 2 = yx - y - x + 2 = y \boxtimes x
    \end{equation*}
    Therefore, $(\Z, \boxplus, \boxtimes)$ is a commutative ring.

    (ii) We claim that $(\Z, \boxplus, \boxtimes)$ is a ring with identity. To see this, let $x \in \Z$ be arbitrary. Then if such a $y \in \Z$ exists such that $x \boxtimes y = y \boxtimes x = x$, then
    \begin{align*}
        x \boxtimes y &= x \\
        xy - x - y + 2 &= x \\
        xy - 2x - y + 2 &= 0 \\
        x(y - 2) - (y - 2) &= 0 \\
        (x - 1)(y - 2) &= 0
    \end{align*}
    So from the above equation, we have that $y = 2$ is the identity over $\boxtimes$. Then observe that $2 \neq 0$, and for any $x \in \Z$,
    \begin{align*}
        x \boxtimes 2 &= x2 - x - 2 + 2 = x
    \end{align*}
    and
    \begin{equation*}
        2 \boxtimes x = 2x - 2 - x + 2 = x
    \end{equation*}
    as desired.

    (iii) We claim that $(\Z, \boxplus, \boxtimes)$ is an integral domain. To see this, note that from (i) and (ii) it is a commutative ring with identity. Assume that for $x, y \in \Z$ such that $x \boxtimes y = 0$. Then
    \begin{align*}
        x \boxtimes y &= xy - x - y + 2 = 0
    \end{align*}
    But then the equation above can be expressed as $y = \frac{x - 2}{x - 1}$ with $x \neq 1$, in which by some brief analysis, has a zero at $x = 2$, which would then yield that $y = 0$. So we have that $x \neq 2$ but $y = 0$. On the other hand, we can also express the equation as $x = \frac{y - 2}{y - 1}$ with $y \neq 1$, and similarly, we would obtain that $x = 0$ but $2 \neq 0$. Therefore, we have shown that either $x = 0$ or $y = 0$.

    (iv) We claim that $(\Z, \boxplus, \boxtimes)$ is a not a field. To see this, note that from (i) and (ii), it is a commutative ring with identity. To show that $(\Z, \boxplus, \boxtimes)$ is a field, we need to show that it is a division ring. So let $x \neq 0\in \Z$ be arbitrary. Then we seek a $y \in \Z$ so that $x \boxtimes y = y \boxtimes x = 2$ (because 2 is the identity over $\boxtimes$). Then solving for $y$,
    \begin{align*}
        xy - x - y + 2 &= 2 \\
        xy - x - y &= 0 \\
        y(x - 1) &= x \\
        y &= \frac{x}{x - 1}
    \end{align*}
    However, note that by taking $x = 3$, we obtain that $y = \frac{3}{2} \notin \Z$, which is absurd. Therefore, $(\Z, \boxplus, \boxtimes)$ cannot be a field.
\end{solution}

\newpage

\begin{question}
    Let $\Z_n[i] = \{a + ib : a, b \in \Z_n, i^2 = -1\}$ denote the Gaussian integers modulo $n$.
    \begin{itemize}
        \item[(a)] Generate the multiplication table of $\Z_n[i]$ for $n = 2, 3, ..., 7$. (Use computer. Submit the tables for only $n = 2, 3$.)
        \item[(b)] Determine all integers $n \geq 2$ for which $\Z_n[i]$ is an integral domain, hence, a field. (Hint: Fermat's theorem on the sum of squares. You may assume that $a^2 + b^2 = 0 \bmod p$ implies $a = b = 0 \bmod p$.)
    \end{itemize}
\end{question}

\begin{solution}
    (a) We have the following tables generated for $n = 2$ and $n = 3$ as follows:
    \begin{center}
        \begin{tabular}{|c||c|c|c|c|}
            \hline
            $\cdot_2$ & $0$ & $1$     & $i$     & $1 + i$ \\ \hline\hline
            $0$       & $0$ & $0$     & $0$     & $0$     \\ \hline
            $1$       & $0$ & $1$     & $i$     & $1 + i$ \\ \hline
            $i$       & $0$ & $i$     & $1$     & $1 + i$ \\ \hline
            $1 + i$   & $0$ & $1 + i$ & $1 + i$ & $0$     \\ \hline
            \end{tabular}
    \end{center}

    \begin{table}[h]
        \centering
        \resizebox{\columnwidth}{!}{%
        \begin{tabular}{|c||c|c|c|c|c|c|c|c|c|}
        \hline
        $\cdot_3$ & 0 & 1        & 2        & $i$      & $2i$     & $1 + i$  & $1 + 2i$ & $2 + i$  & $2 + 2i$ \\ \hline\hline
        0         & 0 & 0        & 0        & 0        & 0        & 0        & 0        & 0        & 0        \\ \hline
        1         & 0 & 1        & 2        & $i$      & $2i$     & $1 + i$  & $1 + 2i$ & $2 + i$  & $2 + 2i$ \\ \hline
        2         & 0 & 2        & 1        & $2i$     & $i$      & $2 + 2i$ & $2 + i$  & $1 + 2i$ & $1 + i$  \\ \hline
        $i$       & 0 & $i$      & $2i$     & 2        & 1        & $2 + i$  & $1 + i$  & $2 + 2i$ & $1 + 2i$ \\ \hline
        $2i$      & 0 & $2i$     & $i$      & 1        & 2        & $1 + 2i$ & $2 + 2i$ & $1 + i$  & $2 + i$  \\ \hline
        $1 + i$   & 0 & $1 + i$  & $2 + 2i$ & $2 + i$  & $1 + 2i$ & $2i$     & 2        & 1        & $i$      \\ \hline
        $1 + 2i$  & 0 & $1 + 2i$ & $2 + i$  & $1 + i$  & $2 + 2i$ & 2        & $i$      & $2i$     & 1        \\ \hline
        $2 + i$   & 0 & $2 + i$  & $1 + 2i$ & $2 + 2i$ & $1 + i$  & 1        & $2i$     & $i$      & 2        \\ \hline
        $2 + 2i$  & 0 & $2 + 2i$ & $1 + i$  & $1 + 2i$ & $2 + i$  & $i$      & 1        & 2        & $2i$     \\ \hline
        \end{tabular}%
        }
        \end{table}

    (b) We will break down the proof into three cases.

    \underline{Case 1: (If $n$ is an even integer)} First consider the case when $n = 2$. We claim that $\Z_2[i]$ is not an integral domain. To see this, let us take $a = b = 1 + i \in \Z_2[i]$ such that $(1 + i)^2 = 0$. However, $1 + i \neq 0$, so $\Z_2[i]$ cannot be an integral domain. Recall that $\Z_p$ is an integral domain if and only if $p$ is prime. Now if $n = 2k$ for some integer $k \in \Z$, then note that $\Z_{2k}$ would not be an integral domain as well, because $2k$ is a composite number, and therefore, $\Z_{2k}$ cannot be an integral domain. Furthermore, because $\Z_{2k} \subset \Z_{2k}[i]$, then $\Z_{2k}[i]$ cannot be an integral domain as well.

    \underline{Case 2: (If $p = 4k + 1$ is a prime for some $k \in \Z$)} Using Fermat's Theorem of Sum of Squares, then for integers $a, b \in \Z_p$,
    \begin{align*}
        a^2 + b^2 &= p \\
        a^2 + b^2 &= p \\
        (a + ib)(a - ib) &= p
    \end{align*}
    So then this implies that both $a + ib$ and $a - ib$ are zero divisors in $\Z_p[i]$, and thus, $\Z_p[i]$, where $p = 4k + 1$ for some integer $k$, cannot be an integral domain.

    \underline{Case 3: (If $p = 4k + 3$ is a prime for some $k \in \Z$)} We claim that $\Z_p[i]$, \\ where $p = 4k + 3$ for some $k \in \Z$, is an integral domain. If this were not the case, let us assume that $a + ib \in \Z_p[i]$ is a zero divisor. Then for $c + id \in \Z_p[i]$ such that $c + id \neq 0$, we then have that
    \begin{equation*}
        (a + ib)(c + id) = 0
    \end{equation*}
    which is in $\Z_p[i]$. So then,
    \begin{equation*}
        (a + ib)(c + id) = (ac - bd) + i(ad + bc) = 0
    \end{equation*}
    and thus, $(ac - bd) \bmod p$ and $(ad + bc) \bmod p$, or equivalently:
    \begin{align*}
        ac - bd \equiv 0 \pmod p \tag{1} \\
        ad + bc \equiv 0 \pmod p \tag{2}
    \end{align*}
    Because $a + ib \in \Z_p[i]$ is a zero divisor, then at least one of $a$ or $b$ is not zero in $\Z_p$. Assume, without loss of generality, that $a \neq 0$. Then using (1)
    \begin{align*}
        ac - bd &\equiv 0 \pmod p \\
        ac &\equiv bd \pmod p \\
        (ac)a^{-1} &\equiv (bd)a^{-1} \pmod p \\
        (aa^{-1})c &\equiv bda^{-1} \pmod p \\
        c &\equiv bda^{-1} \pmod p
    \end{align*}
    Then substituting to (2), we obtain
    \begin{align*}
        ad + bc &\equiv 0 \pmod p \\
        ad + b(bda^{-1}) &\equiv 0 \pmod p \\
        ad + b^2da^{-1} &\equiv 0 \pmod p
    \end{align*}
    Note that $d \neq 0$ in this case. Otherwise, $c + id = 0$, which is absurd. Thus, note that
    \begin{align*}
        ad + b^2da^{-1} &\equiv 0 \pmod p \\
        d(a + b^2a^{-1}) &\equiv 0 \pmod p \\
        a + b^2a^{-1} &\equiv 0 \pmod p \\
        (a + b^2a^{-1})a^{-1} &\equiv 0 \pmod p \\
        1 + b^2(a^{-1})^2 &\equiv 0 \pmod p \\
        (ba^{-1})^2 \equiv -1 \pmod p
    \end{align*}
    But then from here, because $p = 4k + 3$ is a prime for some $k \in \Z$, then we note that the last equation $(ba^{-1})^2 \equiv -1 \pmod p$ has no solutions, since $-1$ is not a quadratic residue mod $p$. Therefore, no such zero divisors exists when $p = 4k + 3$ is a prime for some $k \in \Z$. Furthermore, since we have that $\Z_p \subset \Z_p[i]$ and since $\Z_p$ is a field, then we also obtain that $\Z_p[i]$ is a field.
\end{solution}

\newpage

\begin{question}
    Let $R$ be a ring. Define the \emph{center of $R$} to be
    \begin{equation*}
        Z(R) = \{a \in R : ar = ra \ \text{for all $r \in R$}\}
    \end{equation*}
    Prove that $Z(R)$ is a commutative subring of $R$.
\end{question}

\begin{solution}
    To show that $Z(R) \leq R$\footnote{This notation is similar to saying that $H$ is a subgroup of $G$, or $H \leq G$. So if $S$ is a subring of $R$, we will use the notation $S \leq R$.} is a commutative subring, we need to verify the following properties:
    \begin{enumerate}
        \item Clearly, $Z(R) \neq \emptyset$ because $0 \in Z(R)$, which implies that $0 \in R$ as well, but satisfies the condition that $0r = r0 = 0$, which is true for any $r \in R$.
        \item Let $a, b \in Z(R)$ be arbitrary. We need to show that $ab \in Z(R)$. Indeed, if $a, b \in Z(R)$, then observe that for any $r \in R$,
        \begin{equation*}
            (ab)r = a(br) = a(rb) = (ar)b = (ra)b = r(ab)
        \end{equation*}
        by using the associative property of $R$.
        \item Let $a, b \in Z(R)$ be arbitrary. We need to show that $a - b \in Z(R)$. Indeed, if $a, b \in Z(R)$, then for any $r \in R$,
        \begin{equation*}
            (a - b)r = ar - br = ra - rb = r(a - b)
        \end{equation*}
        by using the distributive property of the ring $R$.
        \item Finally, note that $Z(R)$ is commutative by its definition, because for any $a \in R$, $ar = ra$ for any $r \in R$, so $Z(R)$ is commutative.
    \end{enumerate}
    Therefore, we have shown that $Z(R) \leq R$.
\end{solution}

\newpage

\begin{question}
    An element $a$ is an \emph{idempotent} if $a^2 = a$.
    \begin{itemize}
        \item[(a)] Prove that the only idempotents in an integral domain are 0 and 1.
        \item[(b)] Find a ring with an idempotent that is not equal to 0 nor 1.
        \item[(c)] Let $R$ be a commutative ring with characteristic 2. Prove that the set $S = \{a \in R : a^2 = a\}$ is a subring of $R$.
    \end{itemize}
\end{question}

\begin{solution}
    (a) Let $R$ be an integral domain, and let $a \in R$. Assume that $a^2 = a$. Then $a^2 - a = 0$, and so by factoring, $a(a - 1) = 0$, so either $a = 0$ or $a - 1 = 0$ (thus, $a = 1$). Therefore, $a = 0$ and $a = 1$ are the only idempotents in the integral domain.

    (b) Take $\Z_{15} = \{0, 1, 2, 3, 4, 5, 6, 7, 8, 9, 10, 11, 12, 13, 14\}$. Then we have
    \begin{multicols}{3}
        \begin{itemize}
            \item \textcolor{red}{$0^2 = 0$}
            \item \textcolor{red}{$1^2 = 1$}
            \item $2^2 = 4$
            \item $3^2 = 9$
            \item $4^2 = 16 \equiv 1$
            \item $5^2 = 25 \equiv 10$
            \item \textcolor{red}{$6^2 = 36 \equiv 6$}
            \item $7^2 = 49 \equiv 4$
            \item $8^2 = 64 \equiv 4$
            \item $9^2 = 81 \equiv 6$
            \item \textcolor{red}{$10^2 = 100 \equiv 10$}
            \item $11^2 = 121 \equiv 1$
            \item $12^2 = 144 \equiv 9$
            \item $13^2 = 169 \equiv 4$
            \item $14^2 = 196 \equiv 1$
        \end{itemize}
    \end{multicols}
    Here, we observe that $6^2 = 6$ and $10^2 = 10$ are idempotents in $\Z_{15}$.

    (c) To show that $S \leq R$, we need to verify the following properties.
    \begin{enumerate}
        \item $S \neq \emptyset$ because $0, 1 \in S$ means that $0^2 = 0$ and $1^2 = 1$.
        \item Let $a, b \in S$ be arbitrary. We need to show that $ab \in S$. If $a, b \in S$, then
        \begin{equation*}
            (ab)^2 = a^2b^2 = ab
        \end{equation*}
        \item Let $a, b \in S$ be arbitrary. We need to show that $a + b \in S$ (rather than showing $a - b \in S$). If $a, b \in S$, then $a^2 = a$ and $b^2 = b$. Furthermore, because $R$ has characteristic 2, then $2ab = 0$. Therefore,
        \begin{equation*}
            (a + b)^2 = a^2 + 2ab + b^2 = a^2 + b^2 = a + b
        \end{equation*}
    \end{enumerate}
    Therefore, we have shown that $S \leq R$.
\end{solution}

\newpage

\begin{question}
    \begin{itemize}
        \item[(a)] Let $R$ be a commutative ring with identity. Show that the set of units in $R$, $U(R)$, is an abelian group under $\times_R$.
        \item[(b)] Let $F$ be a finite field with $n$ elements. Show that $a^{n - 1} = 1$ for all $a \neq 0 \in F$.
    \end{itemize}
\end{question}

\begin{solution}
    (a) Recall that $U(R) = \{n \in R : \gcd(k, n) = 1\}$. We show that $(U(R), \times_R)$ is an abelian group as follows:
    \begin{enumerate}
        \item (Associativity) Since $R$ is a commutative ring with identity, then $R$ is also associative, and therefore, $U(R)$ is also associative, as it is inherited from $R$.
        \item (Identity) The element $1 \in U(R)$ because $1 \in R$, so for any $x \in U(R)$
        \begin{equation*}
            1 \times_R a = a \quad a \times_R 1 = a
        \end{equation*}
        \item (Inverse) Let $a \in U(R)$ be arbitrary. Then because $R$ is a commutative ring with identity, then there exists an element $b \in R$ such that
        \begin{equation*}
            a \times_R b = b \times_R a = 1
        \end{equation*}
        so $b \in U(R)$ as well, so an inverse exists.
        \item (Abelian) Since $R$ is a commutative ring with identity, then $U(R)$ is also an abelian group under $\times_R$.
    \end{enumerate}

    (b) Let $a \neq 0 \in F$ be arbitrary. Since $F$ is a field with $n$ elements, then it is a multiplicative group of order $n - 1$. Then the order of $a$ must be a divisor of $n - 1$. Assume that the order of $a$ is $m \in \N$, then because $a$ is a divisor of $n - 1$, we have $m \mid n - 1$, so there exists an integer $b \in \N$ such that $bm = n - 1$, and so
    \begin{equation*}
        a^{n - 1} = b^m = (a^m)^b = 1^b = 1
    \end{equation*}
    So $a^{n - 1} = 1$ for every $a \neq 0 \in F$, as desired.
\end{solution}

\newpage

\begin{question}
    Let $F$ be a field and let $K$ be a subset of $F$ with at least two elements. Prove that $K$ is a subfield of $F$ if for any $a, b \in K$, $a - b \in K$ and $ab^{-1} \in K$.
\end{question}

\begin{solution}
    Let $a, b \in K$ be arbitrary. Then we need to show that $a - b \in K$ and $ab^{-1} \in K$ (assuming that $b \neq 0$). First, note that if $a \in K$, then $a - a = 0 \in K$ and also $-a = 0 - a \in K$. Then if $a, b \in K$ are arbitrary, then $-b \in K$ as well, and so $a + (-b) = a - b \in K$, as required. Now let $b \in K$ such that $b \neq 0$. Since $K$ contains at least two elements, and because $F$ is a field, then $b \cdot b^{-1} = 1 \in K$, and also $b^{-1} = 1 \cdot b^{-1} \in K$ as well. Now let $a, b \in K$ be arbitrary such that $b \neq 0$. Then because $b \neq 0$ and $F$ is a field, then $b^{-1}$ exists in $K$, and $ab^{-1} \in K$. Therefore, we have shown that $a - b \in K$ and $ab^{-1} \in K$, so $K \leq F$, as required.
\end{solution}

\newpage

\begin{question}
    \begin{itemize}
        \item[(a)] Let $R$ be a commutative ring with prime characteristic $p$. Show that for $r, s \in R$,
        \begin{itemize}
            \item[(i)] $(r + s)^p = r^p + s^p$
            \item[(ii)] $(r + s)^{p^m} = r^{p^m} + s^{p^m}$ for all positive integers $m$.
            \item[(iii)] The Frobenius map $x \mapsto x^p$ is a ring homomorphism from $R$ to $R$.
        \end{itemize}
        \item[(b)] Give an example of a ring of characteristic 4 and elements $r$ and $s$ such that $(r + s)^4 \neq r^4 + s^4$.
    \end{itemize}
\end{question}

\begin{solution}
    (a) We first require the following lemma in order to prove (i).

    \begin{center}
        \fbox{\fbox{\parbox{0.9\linewidth}{\begin{claim}
            If $p \geq 2$ is prime, then $p \mid {p \choose k}$ for $0 \leq k \leq p$
        \end{claim}}}}
    \end{center}

    Given the claim, for $p$ prime and $r, s \in R$ where $R$ is a commutative ring with prime characteristic, we have by the binomial theorem,
    \begin{align*}
        (r + s)^p &= \sum_{k = 0}^{p} {p \choose k} r^{p - k}s^k \\
        &= {p \choose 0} r^p + \sum_{k = 1}^{p - 1} {p - 1 \choose k} r^{p - k}s^k + {p \choose p} s^p \\
        &= r^p + s^p + \sum_{k = 1}^{p - 1} {p \choose k} r^{p- k}s^k
    \end{align*}
    Then, since $p \mid {p \choose k}$, then it follows that $p \mid \sum_{k = 0}^{p - 1} {p \choose k}r^{p - k} s^k$, so there exists an integer $m \in \Z$ such that
    \begin{equation*}
        \sum_{k = 1}^{p - 1} {p \choose k}r^{p - k}s^k = mp
    \end{equation*}
    Furthermore, since $R$ is a commutative ring with characteristic $p$, then
    \begin{equation*}
        \sum_{k = 1}^{p - 1} {p \choose k}r^{p - k}s^k = 0
    \end{equation*}
    Hence,
    \begin{equation*}
        (r + s)^p = r^p + s^p
    \end{equation*}
    as desired.

    Now to prove the claim, observe that
    \begin{equation*}
        p(p - 1)! = {p \choose k} k!(p - k)!
    \end{equation*}
    which then implies that $p \mid {p \choose k} k!(p - k)!$ and furthermore, because $p$ is prime, we have either $p \mid {p \choose k}$, $p \mid k!$, or $p \mid (p - k)!$. However, $p \nmid k!$ and $p \nmid (p - k)!$. To see this, assume otherwise that $p \mid k!$. Since $p$ is prime, then $p \mid i$ for some $1 \leq i \leq k$. But then $1 \leq p \leq i$, which is absurd, because $1 \leq k \leq p$. Similarly, $1 \leq (p - k) < p$ so $p \nmid (p - k)!$.

    (ii) In a similar manner as in part (i), we use the binomial theorem to see that
    \begin{equation*}
        (r + s)^{p^m} = r^{p^m} + \sum_{k = 1}^{p^m - 1} {p^m \choose k} r^{p^m - k}s^k + s^{p^m}
    \end{equation*}
    Now we just need to show that $\sum_{k = 1}^{p^m - 1} {p^m \choose k} r^{p^m - k}s^k = 0$. However, note that $p^m \mid \sum_{k = 0}^{p^m - 1} {p^m \choose k} r^{p^m - k}s^k$ for each $1 \leq k \leq p^m - 1$, so each of these coefficients are also divisible by $p$. Therefore, there is some $n \in \Z$ such that $\sum_{k = 0}^{p^m - 1} {p^m \choose k} r^{p^m - k}s^k = np$, and thus,
    \begin{align*}
        (r + s)^{p^m} &= r^{p^m} + \sum_{k = 1}^{p^m - 1} {p^m \choose k} r^{p^m - k}s^k + s^{p^m} \\
        &= r^{p^m} + np + s^{p^m} \\
        &= r^{p^m} + 0 + s^{p^m} \\
        &= r^{p^m} + s^{p^m}
    \end{align*}
    as required.

    (iii) Let $\phi : R \to R$ be a mapping defined by $\phi(x) = x^p$, where $p$ is prime. We want to show that $\phi$ is a ring homomorphism, i.e. for all $x, y \in R$,
    \begin{equation*}
        \phi(x + y) = \phi(x) + \phi(y) \quad \phi(xy) = \phi(x)\phi(y)
    \end{equation*}
    To show the addition, observe that
    \begin{align*}
        \phi(x + y) &= (x + y)^p \\
        &= x^p + y^p \tag{from (i)} \\
        &= \phi(x) + \phi(y)
    \end{align*}
    To show the multiplication, observe that
    \begin{align*}
        \phi(xy) &= (xy)^p \\
        &= x^p y^p \tag{because $R$ is commutative} \\
        &= \phi(x)\phi(y)
    \end{align*}
    Therefore, we have shown that $\phi$ is a ring homomorphism.
\end{solution}

\newpage

\begin{question}
    Let $\phi : R \to S$ be a ring homomorphism. Let $\phi(R) = \{\phi(r) : r \in R\}$. Prove each of the following statements:
    \begin{itemize}
        \item[(a)] $\phi(0_R) = 0_S$
        \item[(b)] $\phi(-b) = -\phi(b)$ for all $b \in R$
        \item[(c)] $\phi(R)$ is a subring of $S$
        \item[(d)] If $R$ is a commutative subring, then $\phi(R)$ is a commutative subring.
        \item[(e)] Suppose $R$ and $S$ are rings with identities. If $\phi$ is onto, then $\phi(1_R) = 1_S$.
        \item[(f)] If $R$ is a field and $\phi(R) \neq \{0_S\}$ then $\phi(R)$ is a field.
    \end{itemize}
\end{question}

\begin{solution}
    (a) To show that $\phi(0_R) = 0_S$, observe that
    \begin{align*}
        \phi(0_R) &= \phi(0_R + 0_R) \\
        &= \phi(0_R) + \phi(0_R)
    \end{align*}
    Since $S$ is a ring, then $\phi(0_R)$ has an additive inverse, namely $-\phi(0_R)$, and therefore, applying $-\phi(0_R)$ to both sides of the equation yields
    \begin{align*}
        \phi(0_R) - \phi(0_R) &= \phi(0_R) + \phi(0_R) - \phi(0_R) \\
        0_S &= \phi(0_R)
    \end{align*}
    as desired.

    (b) Since $0_R = b + (-b)$, then $\phi(0_R) = \phi(b + (-b)) = \phi(b) + \phi(-b)$. Since $\phi(0_R) = 0_S$, then $0_S = \phi(b) + \phi(-b)$, and thus, $\phi(-b) = -\phi(b)$, as required.

    (c) To show that $\phi(R) \leq S$, we need to verify the three properties:
    \begin{enumerate}
        \item Here, $\phi(R) \neq \emptyset$ because $0_R \in R$, and $\phi(0_R) = 0_S$.
        \item Let $\phi(x), \phi(y) \in \phi(R)$ be arbitrary, and hence, $-\phi(y) \in \phi(R)$. Then we show that $\phi(x) - \phi(y) \in \phi(R)$. Since $\phi$ is a homomorphism, then
        \begin{equation*}
            \phi(x + (-y)) = \phi(x) + \phi(-y) = \phi(x) - \phi(y) \in \phi(R)
        \end{equation*}
        \item Let $\phi(x), \phi(y) \in \phi(R)$ be arbitrary. Then we show that $\phi(x)\phi(y) \in \phi(R)$. Since $\phi$ is a homomorphism,
        \begin{equation*}
            \phi(xy) = \phi(x)\phi(y) \in \phi(R)
        \end{equation*}
    \end{enumerate}
    Therefore, we have shown that $\phi(R) \leq S$, as desired.

    (d) To show that $\phi(R)$ is a commutative subring, let $x, y \in R$ be arbitrary. Then $\phi(x), \phi(y) \in \phi(R)$, and since $R$ is a commutative subring, and $\phi$ is a homomorphism,
    \begin{align*}
        \phi(x)\phi(y) &= \phi(xy) \\
        &= \phi(yx) \\
        &= \phi(y)\phi(x)
    \end{align*}
    as required.

    (e) Let $a = \phi(1_R)$ and let $r \in R$ be such that $\phi(r) = 1_S$ (because $\phi$ is onto, such an $r$ exists). Then, because $\phi$ is a homomorphism
    \begin{align*}
        1_S &= \phi(r) \\
        &= \phi(1_R \cdot r) \\
        &= \phi(1_R) \cdot \phi(r) \\
        &= a\phi(r) \\
        &= a \cdot 1_S \\
        &= a
    \end{align*}
    Therefore, $a = \phi(1_R) = 1_S$, as required.

    (f) Note that because $R$ is a field, then $R$ is a commutative divison ring, so $\phi(R)$ is also commutative from (d). Now assume that $\phi(1) = 0$. Then for $r \in R$,
    \begin{equation*}
        \phi(r) = \phi(r)\phi(1) = 0
    \end{equation*}
    which is absurd because it would contradict that $\phi$ is not the zero function. Therefore, $\phi(1) \neq 0$. Now let $\phi(r) \in \phi(R)$ be such that $\phi(r) \neq 0$. Since $\phi(r) \neq 0$, then $r \neq 0$. But then $r$ has an inverse $r^{-1}$, because $R$ is a field, so then
    \begin{equation*}
        \phi(1) = \phi(r \cdot r^{-1}) = \phi(r) \phi(r^{-1})
    \end{equation*}
    Therefore, $\phi(r)$ has an inverse in $\phi(R)$, and so $\phi(R)$ is a field.
\end{solution}

\newpage

\begin{question}
    Consider the ring $S = \left\{\begin{bmatrix} a & b \\ -b & a \end{bmatrix} : a, b \in \R\right\}$ with matrix addition and matrix multiplication. Show that $\phi : \C \to S$ given by
    \begin{equation*}
        \phi(a + ib) = \begin{bmatrix} a & b \\ -b & a \end{bmatrix}
    \end{equation*}
    is a ring isomorphism.
\end{question}

\begin{solution}
    To show that $\phi$ is a ring isomorphism, we check the following.
    \begin{enumerate}
        \item One-to-One: If $\phi(a + ib) = \phi(c + id)$, then $\begin{bmatrix} a & b \\ -b & a \end{bmatrix} = \begin{bmatrix} c & d \\ -d & c \end{bmatrix}$, so $a = c$ and $b = d$, so $a + ib = c + id$.
        \item Onto: Because $\phi$ is an injection, if $\phi(a + ib) = \begin{bmatrix} 0 & 0 \\ 0 & 0 \end{bmatrix}$, then $a = 0$ and $b = 0$ and so $\phi$ is onto.
        \item Let $a + ib, c + id \in \C$. Then observe that
        \begin{align*}
            \phi((a + ib) + (c + id)) &= \phi((a + c) + i(b + d)) \\
            &= \begin{bmatrix} a + c & b + d \\
                -(b + d) & a + c
            \end{bmatrix} \\
            &= \begin{bmatrix} a & b \\ -b & a \end{bmatrix} + \begin{bmatrix} c & d \\ -d & c \end{bmatrix} \\
            &= \phi(a + ib) + \phi(c + id)
        \end{align*}
    \end{enumerate}
    \item Let $a + ib, c + id \in \C$, then observe that
    \begin{align*}
        \phi((a + ib)(c + id)) &= \phi((ac - bd) + i(ad + bc)) \\
        &= \begin{bmatrix} ac - bd & ad + bc \\ -(ad + bc) & ac - bd \end{bmatrix} \\
        &= \begin{bmatrix} a & b \\ -b & a \end{bmatrix}\begin{bmatrix} c & d \\ -d & c \end{bmatrix} \\
        &= \phi(a + ib)\phi(c + id)
    \end{align*}
    Therefore, we have shown that $\phi : \C \to S$ is a ring isomorphism.
\end{solution}

\newpage

\begin{question}
    Show that $\Z_3[i]$ is ring isomorphic to $\Z_3[x]/\langle{x^2 + 1}\rangle$.
\end{question}

\begin{solution}
    To show that $\Z_3[i] \simeq \Z_3[x]/\langle{x^2 + 1}\rangle$, let $\phi : \Z_3[x] \to \Z_3[i]$ be the mapping defined by $\phi(a + bx) = a + bi$. Then
    \begin{equation*}
        \ker(\phi) = \{a + bx : \phi(a + bx) = 0\} = \langle{x^2 + 1}\rangle
    \end{equation*}
    because for any polynomial such that $p(i) = 0$ has $i$ as a root, and therefore $-i$ because it has integer coefficients and is divisible by $x^2 + 1$ so we can factor and we would obtain the isomorphism as required.
\end{solution}

\end{document}