\documentclass[11pt]{amsart}
\usepackage[utf8]{inputenc}
\usepackage[margin=1in]{geometry}
\usepackage{amsfonts, amssymb, amsmath, amsthm, booktabs, hyperref, pgfplots, tikz, xcolor, multicol, cancel, polynom}

\theoremstyle{}\newtheorem{question}{Question}
\theoremstyle{}\newtheorem*{bonus}{Bonus}
\theoremstyle{definition}\newtheorem*{solution}{Solution}

\newcommand{\K}{\mathbb{K}}
\newcommand{\C}{\mathbb{C}}
\newcommand{\R}{\mathbb{R}}
\newcommand{\Q}{\mathbb{Q}}
\newcommand{\Z}{\mathbb{Z}}
\newcommand{\N}{\mathbb{N}}

\title{MATH 3022 Algebra II \\ Assignment 2}
\author{Joe Tran}
\date{\today}

\begin{document}

\maketitle

\setcounter{question}{1}

\begin{question}
    \begin{itemize}
        \item[(a)] Let $R$ be a commutative ring with unity ($1_R \neq 0_R$). Show that if $\{0_R\}$ and $R$ and the only ideals of $R$, then $R$ is a field.
        \item[(b)] Let $F$ be a field. Use (a) to show that $F[x]$ is not a field.
    \end{itemize}
\end{question}

\begin{solution}
    (a) Suppose that $R$ is not a field. Then there exists some $x \neq 0 \in R$ such that $x$ has no inverse, so $\langle{x}\rangle = \{rx : r \in R\} \neq \langle{0}\rangle$ and we cannot obtain $1_R$ (as this would mean that there exists an $r \in R$ such that $rx = 1$, and since $R$ is commutative, then we also have that $rx = xr = 1$ which implies that $x^{-1} = r$), and it cannot be $R$ also. By contrapositive, we have shown that $R$ is not a field.
    
    (b) Let $p(x) \in F[x]$. Then $xp(x) \in F[x]$ and suppose that $xp(x) = 1$. Then when $x = 0 \in F$, we obtain that $0 = 1 \in F$, which is absurd.
\end{solution}

\begin{question}
    Show that the principal ideal $\langle{x - 1}\rangle$ in $\Z[x]$ is prime but not maximal.
\end{question}

\begin{solution}
    Note that because we have $\Z[x]/\langle{x - 1}\rangle \simeq \Z$, and because $\Z$ is an integral domain, then so is $\Z[x]/\langle{x - 1}\rangle$, and so $\langle{x - 1}\rangle$ is a prime ideal. However, $\Z[x]/\langle{x - 1}\rangle \simeq \Z$ and $\Z$ is not a field, so $\Z[x]/\langle{x - 1}\rangle$ cannot be a field, so $\langle{x - 1}\rangle$ cannot be maximal.
\end{solution}

\setcounter{question}{4}

\begin{question}
    Let $R$ be an integral domain. Assume that the division algorithm always holds in $R[x]$. Prove that $R$ is a field.
\end{question}

\begin{solution}
    To see that $R$ is a field, we need to show that every element in $R$ has a multiplicative inverse. Indeed, let $r \neq 0 \in R$, and let $r_1(x) \in R[x]$ be such that $r_1(x) = r$ and $\deg(r_1(x)) = 0$. Let $p(x) \in R[x]$ be an irreducible polynomial with $\deg(p(x)) \geq 1$. Then by assumption, we use the division algorithm so that
    \begin{equation*}
        r_1(x) = p(x)q(x) + r(x)
    \end{equation*}
    where either $r(x) = 0$ or $\deg(r(x)) < \deg(p(x))$. Now we consider the following cases:
    \begin{itemize}
        \item Case 1: If $r(x) = 0$, then $r_1(x) = p(x)q(x)$, but $r_1$ would be a multiple of $p(x)$, which is absurd, because $r \neq 0 \in R$ and $\deg(r_1(x)) = 0$, while $\deg(p(x)) > 0$.
        \item Case 2: If $0 = \deg(r(x)) < \deg(p(x))$ and since $\deg(r(x))$ cannot be smaller than 0, so this case does not hold.
    \end{itemize}
    Therefore, there is no such polynomials of degree greater than 0 in $R[x]$.

    Now consider the polynomial given by
    \begin{equation*}
        p(x) = xr + 1
    \end{equation*}
    Since there are no irreducible polynomials of degree 0, then $p(x)$ is reducible. Then by the division algorithm, there exists polynomials $a(x)$ and $b(x)$ such that
    \begin{equation*}
        p(x) = a(x)b(x) = xr + 1
    \end{equation*}
    which implies that $xr$ is a multiple of an element of $R$ and so, there exists an element $r^{-1}$ such that $rr^{-1} = 1$. Therefore, as $r \neq 0\in R$ was arbitrary, every nonzero element in $R$ has a multiplicative inverse, so $R$ is a field.
\end{solution}

\setcounter{question}{7}

\begin{question}
    Let $p$ be prime.
    \begin{itemize}
        \item[(a)] Show that there are $\frac{p(p + 1)}{2}$ reducible polynomials over $\Z_p$ of the form $x^2 + ax + b$.
        \item[(b)] Determine the number of irreducible polynomials over $\Z_p$ of the form $x^2 + ax + b$.
        \item[(c)] Show that there exists a field of order $p^2$ for every prime $p$.
        \item[(d)] Construct a finite field with four elements. Give the addition table and the multiplication table of your field.
    \end{itemize}
\end{question}

\begin{solution}
    (a) Assume that $x^2 + ax + b$ is a reducible polynomial over $\Z_p$. Then there exists $x - \alpha$ and $x - \beta \in \Z_p[x]$ such that
    \begin{equation*}
        x^2 + ax + b = \begin{cases}
            (x - \alpha)(x - \beta) & \text{if $\alpha \neq \beta$} \\
            (x - \alpha)^2 & \text{if $\alpha = \beta$}
        \end{cases}
    \end{equation*}
    Then since over $\Z_p$, we have $|\Z_p| = p$, then the number of quadratic monic polynomials is $p^2$. Since there are ${p \choose 2}$ ways of choosing $\alpha$ and $\beta$ in the first case (without repetition), and $p$ ways for the second case. Therefore,
    \begin{align*}
        {p \choose 2} + p &= \frac{p!}{2!(p - 2)!} + p \\
        &= \frac{p(p - 1)(p - 2)!}{2(p - 2)!} + p \\
        &= \frac{p(p - 1)}{2} + p \\
        &= \frac{p(p - 1) + 2p}{2} \\
        &= \frac{p(p - 1 + 2)}{2} \\
        &= \frac{p(p + 1)}{2}
    \end{align*}
    Therefore, there are $\frac{p(p + 1)}{2}$ reducible polynomials over $\Z_p$ of the form $x^2 + ax + b$.

    (b) Because the number of quadratic monic polynomials is $p^2$ and the number of reducible quadratic monic polynomials is $\frac{p(p + 1)}{2}$ from (a), then the number of irreducible monic quadratic polynomials are
    \begin{equation*}
        p^2 - \frac{p(p + 1)}{2} = \frac{2p^2 - p^2 - p}{2} = \frac{p^2 - p}{2} = \frac{p(p - 1)}{2}
    \end{equation*}

    (c) Since there is a polynomial of the form $x^2 + ax + b$ that is irreducible over $\Z_p$, then the quotient $\Z_p[x]/\langle{x^2 + ax + b}\rangle$ is a field with $p^2$ elements, since as mentioned from (a), the number of quadratic monic polynomials is $p^2$.

    (d) First let us consider $\Z_2$. We seek an irreducible polynomial $p(x)$ of degree 1 over $\Z_2$. Note that the following polynomials of degree 1 are possible in $\Z_2$:
    \begin{equation*}
        p(x) = x \quad p(x) = x + 1
    \end{equation*}
    However, note that $p(x) = x + 1$ is irreducible since $f(0) = 1 \neq 0$, $f(1) = 2 \neq 0$, and therefore, $\Z_2[x]/\langle{x + 1}\rangle$ is a finite field of order 4. Note that if $I = \langle{x + 1}\rangle$, then the quotient ring is given as
    \begin{equation*}
        \Z_2[x]/\langle{x + 1}\rangle = \{0 + \langle{x + 1}\rangle, 1 + \langle{x + 1}\rangle, x + \langle{x + 1}\rangle, x + 1 + \langle{x + 1}\rangle\}
    \end{equation*}
    Then our addition table is given as
    \begin{center}
        \begin{tabular}{|c||c|c|c|c|} \hline
            $+$ & $0 + I$ & $1 + I$ & $x + I$ & $(x + 1) + I$ \\ \hline
            $0 + I$ & $0 + I$ & $1 + I$ & $x + I$ & $(x + 1) + I$ \\ \hline
            $1 + I$ & $1 + I$ & $0 + I$ & $(x + 1) + I$ & $x + I$ \\ \hline
            $x + I$ & $x + I$ & $(x + 1) + I$ & $0 + I$ & $1 + I$ \\ \hline
            $(x + 1) + I$ & $(x + 1) + I$ & $x + I$ & $1 + I$ & $0 + I$ \\ \hline
        \end{tabular}
    \end{center}
    and the multiplication table is given as
    \begin{center}
        \begin{tabular}{|c||c|c|c|c|} \hline
            $\cdot$ & $0 + I$ & $1 + I$ & $x + I$ & $(x + 1) + I$ \\ \hline
            $0 + I$ & $0 + I$ & $0 + I$ & $0 + I$ & $0 + I$ \\ \hline
            $1 + I$ & $0 + I$ & $1 + I$ & $x + I$ & $(x + 1) + I$ \\ \hline
            $x + I$ & $0 + I$ & $x + I$ & $(x + 1) + I$ & $1 + I$ \\ \hline
            $(x + 1) + I$ & $0 + I$ & $(x + 1) + I$ & $1 + I$ & $x + I$ \\ \hline
        \end{tabular}
    \end{center}
\end{solution}

\setcounter{question}{9}

\begin{question}
    Either prove that $f(x) = 3x^5 - 4x^4 + 7x^3 + 16x^2 - 2$ is irreducible over $\Q$, or factor it into a product of irreducible factors in $\Q[x]$.
\end{question}

\begin{solution}
    We claim that $f(x)$ is irreducible over $\Q$. Indeed, say we take $x = -1 \in \Q$. Then observe that
    \begin{equation*}
        f(-1) = 3(-1)^5 - 4(-1)^4 + 7(-1)^3 + 16(-1)^2 - 2 = 0
    \end{equation*}
    so $x + 1 \in \Q[x]$ is a factor of $f(x)$. Then by performing long division,
    \begin{equation*}
        \polylongdiv{3x^5 - 4x^4 + 7x^3 + 16x^2 - 2}{x + 1}
    \end{equation*}
    Now by the division algorithm, we can write
    \begin{equation*}
        3x^5 - 4x^4 + 7x^3 + 16x^2 - 2 = (x + 1)(3x^4 - 7x^3 + 14x^2 + 2x - 2)
    \end{equation*}
    
    Let $g(x) = 3x^4 - 7x^3 + 14x^2 + 2x - 2$. Say we take $x = \frac{1}{3} \in \Q$. Then observe that
    \begin{equation*}
        g\left(\frac{1}{3}\right) = 3\left(\frac{1}{3}\right)^4 - 7\left(\frac{1}{3}\right)^3 + 14\left(\frac{1}{3}\right)^2 + 2\left(\frac{1}{3}\right) - 2 = 0
    \end{equation*}
    so $x - \frac{1}{3} \in \Q[x]$ is a factor of $g(x)$. Then by performing long division,
    \begin{equation*}
        \polylongdiv{3x^4 - 7x^3 + 14x^2 + 2x - 2}{x - \frac{1}{3}}
    \end{equation*}
    Now by the division algorithm,
    \begin{align*}
        3x^4 - 7x^3 + 14x^2 + 2x - 2 &= \left(x - \frac{1}{3}\right)(3x^3 - 6x^2 + 12x + 6) \\
        &= (3x - 1)(x^3 - 2x^2 + 4x + 2)
    \end{align*}

    Let $h(x) = x^3 - 2x^2 + 4x + 2$. We claim that $h(x)$ is irreducible over $\Q$. Indeed, because the leading coefficient of $h(x)$ is 1 and the constant term of $h(x)$ is 2, then we have the test factors of 2, being 1 and 2. Checking each,
    \begin{align*}
        h(1) &= (1)^3 - 2(1)^2 + 4(1) + 2 = 5 \neq 0 \\
        h(2) &= (2)^3 - 2(2)^2 + 4(2) + 2 = 10 \neq 0
    \end{align*}
    Since none of the above test factors are such that $h(x) = 0$, then $h(x)$ is not irreducible.

    Therefore,
    \begin{align*}
        f(x) = (x + 1)(3x - 1)(x^3 - 2x^2 + 4x + 2)
    \end{align*}
\end{solution}

\begin{bonus}
    Complete the questions specified on Page 5 of your Test 1:
    \begin{itemize}
        \item[(2)]
        \begin{itemize}
            \item[(a)] Let $R$ be a ring with identity. Show that if $a \in R$ is a zero divisor, then it is not a unit.
            \item[(b)] Is the converse true? Justify your answer.
        \end{itemize}
    \end{itemize}
\end{bonus}

\begin{solution}
    (a) Assume that $a \in R$ is a zero divisor, and assume for a contradiction that $a \in R$ is a unit. Since $a$ is a zero divisor, then there exists a $b \neq 0 \in R$ such that
    \begin{equation*}
        ab = 0 \tag{1}
    \end{equation*}
    and since $a$ is a unit, then there exists a unique $a^{-1} \in R$ such that
    \begin{equation*}
        aa^{-1} = [a^{-1} a = 1] \tag{2}
    \end{equation*}
    Then right multiplying both sides of (2) in the bracket by $b$ so that
    \begin{align*}
        (a^{-1}a)b &= 1b \\
        a^{-1}(ab) &= b \\
        a^{-1} \cdot 0 &= b \\
        b &= 0
    \end{align*}
    which is absurd because it contradicts the assumption that $b \neq 0 \in R$ and thus contradicts the assumption that $a$ is a zero divisor. Therefore, it must be the case that $a$ cannot be a unit.

    (b) The converse of (a) is if $a$ is not a unit, then $a$ is a zero divisor. We claim that the statement is true. Assume that $a$ is not a unit. Then for every $a^{-1} \in R$, we have that
    \begin{equation*}
        a \cdot a^{-1} = [a^{-1} \cdot a \neq 1] \tag{1}
    \end{equation*}
    And now assume for a contradiction, that $a$ is not a zero divisor. Then for every $b \neq 0 \in R$, we have that $ab \neq 0$, so let $c \neq 0$ be such that
    \begin{equation*}
        ab = c \tag{2}
    \end{equation*}
    Then right multiplying both sides of (1) by $b$ so that
    \begin{align*}
        (a^{-1} \cdot a) \cdot b &\neq 1 \cdot b \\
        a^{-1} \cdot (a \cdot b) &\neq b \\
        a^{-1} \cdot c &\neq b \\
        a \cdot (a^{-1} \cdot c) &\neq a \cdot b \\
        c &\neq ab
    \end{align*}
    which is a contradiction. Therefore, if $a$ is not a unit, then $a$ must be a zero divisor.
\end{solution}

\end{document}