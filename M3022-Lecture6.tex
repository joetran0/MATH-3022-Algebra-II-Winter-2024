\documentclass[11pt]{article}
\usepackage[utf8]{inputenc}
\usepackage[dvipsnames]{xcolor}
\usepackage{amsfonts, amssymb, amsmath, amsthm, booktabs, hyperref, pgfplots, tikz}

\theoremstyle{definition}\newtheorem{definition}{Definition}
\theoremstyle{definition}\newtheorem*{notation}{Notation}
\theoremstyle{definition}\newtheorem{example}{Example}
\theoremstyle{theorem}\newtheorem{theorem}{Theorem}
\theoremstyle{theorem}\newtheorem{corollary}{Corollary}
\theoremstyle{theorem}\newtheorem{proposition}{Proposition}
\theoremstyle{theorem}\newtheorem{lemma}{Lemma}
\theoremstyle{theorem}\newtheorem{question}{Question}
\theoremstyle{remark}\newtheorem{remark}{Remark}

\newcommand{\K}{\mathbb{K}}
\newcommand{\C}{\mathbb{C}}
\newcommand{\CC}{\mathcal{C}}
\newcommand{\R}{\mathbb{R}}
\newcommand{\Q}{\mathbb{Q}}
\newcommand{\Z}{\mathbb{Z}}
\newcommand{\N}{\mathbb{N}}
\newcommand{\F}{\mathbb{F}}
\renewcommand{\SS}{\mathcal{S}}
\newcommand{\T}{\mathcal{T}}
\newcommand{\I}{\mathcal{I}}
\newcommand{\M}{\mathcal{M}}
\renewcommand{\H}{\mathbb{H}}
\newcommand{\teq}{\trianglelefteq}
\renewcommand{\mod}{\ \mathbf{mod} \ }
\DeclareMathOperator{\Char}{char}

\title{MATH 3022 Algebra II: Lecture 3}
\author{Joe Tran}
\date{Winter 2024}

\setlength{\parindent}{0pt}
\setlength{\parskip}{5pt}

\begin{document}

\textbf{MATH 3022 Algebra II} \hfill \textbf{Lecture 6} \\
\textsc{Lecture} \hfill \textsc{Joe Tran}

\begin{example}
    Consider the ring $R = \M_2(\Z_6) = \left\{\begin{bmatrix} a & b \\ c & d \end{bmatrix} : a, b, c, d \in \Z_6\right\}$ and let $A$ be the matrix given by
    \begin{equation*}
        A = \begin{bmatrix} 2 & 4 \\ 0 & 2 \end{bmatrix} \in \M_2(\Z_6)
    \end{equation*}
    Then
    \begin{equation*}
        2\begin{bmatrix} 2 & 4 \\ 0 & 2 \end{bmatrix} = 2A = A + A = \begin{bmatrix} 4 & 2 \\ 0 & 4 \end{bmatrix}
    \end{equation*}
    and also
    \begin{equation*}
        3\begin{bmatrix} 2 & 4 \\ 0 & 2 \end{bmatrix} = 3A = A + A + A = \begin{bmatrix} 0 & 0 \\ 0 & 0 \end{bmatrix}
    \end{equation*}
\end{example}

Recall that the additive order of a ring is the smallest integer $n \in \N$ such that for any $r \in R$, $nr = 0$. If no such integer exists, then the additive order is defined to be 0. From Example 1, we see that $3A = 0$ so smallest integer is 3.

\begin{example}
    From Example 1, say we take the two matrices $2A$ and $3A$ and we add them together. Then
    \begin{equation*}
        2A + 3A = (A + A) + (A + A + A) = 5A = (2 + 3)A
    \end{equation*}
\end{example}

\begin{definition}
    The characteristic of a ring $R$ is the least positive integer $n$ such that $nr = 0$ for any $r \in R$. In other words,
    \begin{equation*}
        \Char(R) = \min\{n \in \N : nr = 0\}
    \end{equation*}
    If no such integer exists, then $nr = 0$.
\end{definition}

\begin{example}
    \begin{itemize}
        \item For the set $\M_2(\Z_6)$ has characteristic 6.
        \item The set $\Z_n$ has characteristic $n$
        \item The set $\Z$ has characteristic 0.
    \end{itemize}
\end{example}

\begin{question}
    Is it possible to have a finite $R$ to have characteristic 0?
\end{question}

Not possible. If $\Char(R) = 0$, then for every $r \in R$, there exists $n \in \N$ with additive order larger than $n$. Then $\{0r, 1r, 2r,..., nr\}$ consists of pairwise distinct elements, so that $|R| > n$. But as $n$ was arbitrary, then $R$ is infinite.

\begin{question}
    Is it possible to have an infinite ring with positive characteristic?
\end{question}

Possible. Take the set $\Z_3[x]$ which is infinite, and it has characteristic 3.

\begin{lemma}
    Let $R$ be a ring with identity $1 \in R$. If $1$ has additive order $\infty$, then $\Char(R) = 0$. If $1$ has additive order $n$, then $\Char(R) = n$.
\end{lemma}

\begin{proof}
    If 1 has $\infty$ additive order, then there is no such positive integer $n$ such that $nr = 0$. Thus, $\Char(R) = 0$. On the other hand, if $n1 = 0$, then for any $r \in R$,
    \begin{equation*}
        nr = r + r + \cdots + r = 1r + 1r + \cdots + 1r = (1 + 1 + \cdots + 1)r = 0r = 0
    \end{equation*}
    Therefore, $\Char(R) = n$.
\end{proof}

\begin{theorem}
    The characteristic of an integral domain is either prime $p$ or 0.
\end{theorem}

\begin{proof}
    Let $D$ be an integral domain. If $D$ is an infinite integral domain, then we are done. On the other hand, if $D$ is an integral domain with characteristic $n$, and suppose that $n = ab$ for some $a, b \in D$. Then by using Lemma 1,
    \begin{equation*}
        n1 = 1 + 1 + \cdots + 1 = \underbrace{(1 + 1 + \cdots + 1)}_{a \ times}\underbrace{(1 + 1 + \cdots + 1)}_{b \ times} = 0
    \end{equation*}
    Then $(a1)(b1) = 0$. Since $D$ is an integral domain that has no zero divisors, then without loss of generality, assume that $a1 = 0$. Then the only possible case in which this could happen is when $a = n$, because $a \leq n$ and $n$ is the smallest positive integer such that $n1 = 0$.
\end{proof}

Observe that the characteristic of a field is either prime $p$ or zero. Indeed, a finite field has positive characteristic.

\begin{example}
    Suppose $R$ is a commutative ring with no zero divisors. Show that all nonzero elements have the same additive order.

    Let $a, b \in R$ be nonzero elements, and suppose that the additive order of $a$ is $m$ and the additive order of $b$ is $n$. Then without loss of generality, let us assume that $n \leq m$. Then
    \begin{equation*}
        n(ab) = \underbrace{ab + ab + \cdots + ab}_{n \ times} = \underbrace{(a + a + \cdots + a)}_{n \ times} b = 0b = 0
    \end{equation*}
    and similarly,
    \begin{equation*}
        n(ab) = \underbrace{ab + ab + \cdots + ab}_{n \ times} = a\underbrace{(b + b + \cdots + b)}_{n \ times} = a0 = 0
    \end{equation*}
    Therefore, $n = m$.
\end{example}

\end{document}