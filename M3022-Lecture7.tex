\documentclass[11pt]{article}
\usepackage[utf8]{inputenc}
\usepackage[dvipsnames]{xcolor}
\usepackage{amsfonts, amssymb, amsmath, amsthm, booktabs, hyperref, pgfplots, tikz}

\theoremstyle{definition}\newtheorem{definition}{Definition}
\theoremstyle{definition}\newtheorem*{notation}{Notation}
\theoremstyle{definition}\newtheorem{example}{Example}
\theoremstyle{theorem}\newtheorem{theorem}{Theorem}
\theoremstyle{theorem}\newtheorem{corollary}{Corollary}
\theoremstyle{theorem}\newtheorem{proposition}{Proposition}
\theoremstyle{theorem}\newtheorem{lemma}{Lemma}
\theoremstyle{theorem}\newtheorem{question}{Question}
\theoremstyle{remark}\newtheorem{remark}{Remark}

\newcommand{\K}{\mathbb{K}}
\newcommand{\C}{\mathbb{C}}
\newcommand{\CC}{\mathcal{C}}
\newcommand{\R}{\mathbb{R}}
\newcommand{\Q}{\mathbb{Q}}
\newcommand{\Z}{\mathbb{Z}}
\newcommand{\N}{\mathbb{N}}
\newcommand{\F}{\mathbb{F}}
\renewcommand{\SS}{\mathcal{S}}
\newcommand{\T}{\mathcal{T}}
\newcommand{\I}{\mathcal{I}}
\newcommand{\M}{\mathcal{M}}
\renewcommand{\H}{\mathbb{H}}
\newcommand{\teq}{\trianglelefteq}
\renewcommand{\mod}{\ \mathbf{mod} \ }
\DeclareMathOperator{\Char}{char}

\title{MATH 3022 Algebra II: Lecture 3}
\author{Joe Tran}
\date{Winter 2024}

\setlength{\parindent}{0pt}
\setlength{\parskip}{5pt}

\begin{document}

\textbf{MATH 3022 Algebra II} \hfill \textbf{Lecture 7} \\
\textsc{Lecture} \hfill \textsc{Joe Tran}

Recall in the previous lecture:
\begin{itemize}
    \item[(1)] The characteristic of an integral domain is either prime or zero.
    \item[(2)] If $R$ has characteristic zero, then $|R|$ is not finite.
    \item[(1) + (2)] The characteristic of a finite field is a prime $p$.
    \item[(3)] If $R$ is a commutative ring with no zero divisors, then every nonzero elements have the same additive order. 
    \item This implies that every nonzero element in an infinite field has additive order $p$.
\end{itemize}

From point •, this would imply that the field $(F, +)$ is a $p$-group.

\begin{definition}
    A group $(G, *)$ is said to be a $p$-group if $|G| = p^m$.
\end{definition}

\begin{corollary}[15.2]
    Let $G$ be a finite group. Then $G$ is a $p$-group if and only if $|G| = p^m$.
\end{corollary}

\begin{corollary}
    Let $F$ be a finite field with characteristic $p$ (prime), then $|F| = p^m$ for some $m \in \Z$.
\end{corollary}

\section{Ring Homomorphisms}

\begin{definition}
    Let $(R, +_R, \cdot_R)$ and $(S, +_S, \cdot_S)$\footnote{I will use triplets to denote the ring $(R, +, \cdot)$, where $R$ is the ring, and $+$ and $\cdot$ are the operations of addition and multiplication, respectively.} be two rings with respect to their own operations. Then the mapping $\phi : R \to S$ is said to be a \emph{ring homomorphism} if for all $x, y \in R$,
    \begin{align*}
        \phi(x +_R y) &= \phi(x) +_S \phi(y) \\
        \phi(x \cdot_R y) &= \phi(x) \cdot_S \phi(y)
    \end{align*}
\end{definition}

Observe that for $n \in \N$ and for all $x \in R$,
\begin{align*}
    \phi(nx) &= \phi(x +_R x +_R \cdots +_R x) \\
    &= \phi(x) +_S \phi(x) +_S + \cdots +_S \phi(x) \\
    &= n\phi(x)
\end{align*}
Also observe that $\phi(x^n) = \phi^n(x)$ and also, $\phi(0_R) = 0_S$\footnote{Question 8 on Assignment 1}.

Taking the first observation and the third observation, if $nx = 0_R$, then $n\phi(x) = 0_S$, and furthermore, if the additive order of $x$ i $n$, then the additive order of $\phi(x) \mid n$.

Also on Assignment: If $R$ and $S$ are rings with identity and $\phi$ is onto, then $\phi(1_R) = 1_S$.

\begin{question}
    Give an example where $\phi$ is not onto and $\phi(1_R) \neq 1_S$.
\end{question}

\begin{example}
    \begin{itemize}
        \item Let $\phi : \Z \to \Z_n$ given by $\phi(x) = x \bmod n$ for all $x \in \Z$ is a ring homomorphism.
        \item Let $\phi : \C \to \C$ be defined by $\phi(a + ib) = a - ib$ for all $a, b \in \R$ is a ring homomorphism.
    \end{itemize}
\end{example}

\begin{definition}
    A ring homomorphism $\phi : R \to S$ is said to be a ring isomorphism if $\phi$ is a bijection.
\end{definition}

\begin{example}
    For a fixed $a \in R$, define $\phi : R[x] \to R$ given by
    \begin{equation*}
        \phi_a(f(x)) = f(a)
    \end{equation*}
    This is called the \emph{evaluation homomorphism}. Indeed,
    \begin{align*}
        \phi_a(f(x) + g(x)) = f(a) + g(x) = \phi_a(f(x)) + \phi_a(g(x))
    \end{align*}
    and
    \begin{equation*}
        \phi_a(f(x)g(x)) = f(a)g(a) = \phi_a(f(x))\phi_a(g(x))
    \end{equation*}
\end{example}

\begin{question}
    Is the mapping $\phi : \R \to \R$ given by $\phi(a) = -a$ a ring homomorphism?
\end{question}

No it is not. Take $2, 3 \in \R$ and observe that
\begin{equation*}
    \phi(2 \cdot 3) = -6 = 6 = -2 \cdot -3 = \phi(2)\phi(3)
\end{equation*}
which is absurd.

\begin{example}
    Determine all ring homomorphisms from $\Z_{12}$ to $\Z_{30}$.

    Let $\phi : \Z_{12} \to \Z_{30}$ be a mapping and let $\phi([x]_{12}) = [y]_{30}$. Then
    \begin{equation*}
        \phi([12]_{12} \cdot [1]_{12}) = 12 \cdot [y]_{30}
    \end{equation*}
    would imply that $12 [y]_{30} = [0]_{30}$, and so $12y \equiv 0 \pmod {30}$, i.e. all $y$ such that $30 \mid 12y$. So $[y]_{30} = 0, 5, 10, 15, 20, 25$.

    Observe that $\phi([1]_{12}) = [y]_{30}$ implies that
    \begin{equation*}
        \phi([x]_{12}) = \phi(\underbrace{1 +_{12} + \cdots +_{12} 1}_{x \ times}) = x ([y]_{30}) = x[y]_{30}
    \end{equation*}
    Thus, it determines the map $\phi$.

    To complete the proof, we need to work with multiplication now. Observe for any $a \in \Z_{12}$,
    \begin{equation*}
        \phi([1]_{12}) = \phi(1 \cdot [1]_{12}) = \phi([1]_{12}) \phi([1]_{12}) = \phi([1]_{12})
    \end{equation*}
    which implies that
    \begin{equation*}
        ([x]_{30})^2 = [x]_{30}
    \end{equation*}
    So we have $[0]_{30}$, $[10]_{30}$, $[15]_{30}$ and $[25]_{30}$.
\end{example}

\end{document}