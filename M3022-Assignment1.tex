\documentclass[11pt]{amsart}
\usepackage[utf8]{inputenc}
\usepackage[dvipsnames]{xcolor}
\usepackage[margin=1in]{geometry}
\usepackage{amsfonts, amssymb, amsmath, amsthm, booktabs, hyperref, pgfplots, tikz, xcolor, mathrsfs}

\theoremstyle{definition}\newtheorem{definition}{Definition}
\theoremstyle{definition}\newtheorem{question}{Question}
\theoremstyle{definition}\newtheorem*{solution}{Solution}
\theoremstyle{definition}\newtheorem{example}{Example}
\theoremstyle{definition}\newtheorem{notation}{Notation}
\theoremstyle{theorem}\newtheorem{theorem}{Theorem}
\theoremstyle{theorem}\newtheorem{corollary}{Corollary}
\theoremstyle{theorem}\newtheorem{lemma}{Lemma}
\theoremstyle{theorem}\newtheorem{proposition}{Proposition}

\newcommand{\A}{\mathcal{A}}
\newcommand{\B}{\mathcal{B}}
\newcommand{\C}{\mathbb{C}}
\newcommand{\CC}{\mathcal{C}}
\newcommand{\D}{\mathcal{D}}
\renewcommand{\d}{\delta}
\newcommand{\E}{\mathcal{E}}
\newcommand{\e}{\varepsilon}
\newcommand{\F}{\mathbb{F}}
\newcommand{\FF}{\mathcal{F}}
\newcommand{\G}{\mathcal{G}}
\renewcommand{\H}{\mathbb{H}}
\newcommand{\I}{\mathcal{I}}
\newcommand{\J}{\mathcal{J}}
\newcommand{\K}{\mathbb{K}}
\renewcommand{\L}{\mathscr{L}}
\newcommand{\M}{\mathcal{M}}
\newcommand{\N}{\mathbb{N}}
\renewcommand{\O}{\mathcal{O}}
\renewcommand{\P}{\mathcal{P}}
\newcommand{\Q}{\mathbb{Q}}
\newcommand{\R}{\mathbb{R}}
\renewcommand{\S}{\mathcal{S}}
\newcommand{\T}{\mathbb{T}}
\newcommand{\U}{\mathcal{U}}
\newcommand{\V}{\mathcal{V}}
\newcommand{\W}{\mathcal{W}}
\newcommand{\X}{\mathcal{X}}
\newcommand{\Y}{\mathcal{Y}}
\newcommand{\Z}{\mathbb{Z}}

\begin{document}

\noindent \textbf{MATH 3022 Algebra II} \hfill \textbf{Assignment 1} \\
\textsc{Assignment} \hfill \textsc{Joe Tran}

\begin{question}\label{question:1}
    We define two operations $\boxplus$ and $\boxtimes$ on $\Z$ as
    \begin{align*}
        a \boxplus b &= a + b - 1 \\
        a \boxtimes b &= ab - a - b + 2
    \end{align*}
    for $a, b \in \Z$.
    \begin{itemize}
        \item[(a)] Show that $\Z$ together with addition $\boxplus$ and multiplication $\boxtimes$ is a ring.
        \item[(b)] Determine if this ring is
        \begin{itemize}
            \item[(i)] A commutative ring
            \item[(ii)] A ring with identity
            \item[(iii)] An integral domain.
            \item[(iv)] A field.
        \end{itemize}
    \end{itemize}
\end{question}

\begin{question}
    Let $\Z_n[i] = \{a + ib : a, b \in \Z_n, i^2 = -1\}$ denote the Gaussian integers modulo $n$.
    \begin{itemize}
        \item[(a)] Generate the multiplication table of $\Z_n[i]$ for $n = 2, 3, ..., 7$. (Use computer. Submit the tables for only $n = 2, 3$.)
        \item[(b)] Determine all integers $n \geq 2$ for which $\Z_n[i]$ is an integral domain, hence, a field. (Hint: Fermat's theorem on the sum of squares. You may assume that $a^2 + b^2 = 0 \bmod p$ implies $a = b = 0 \bmod p$.)
    \end{itemize}
\end{question}

\begin{question}
    Let $R$ be a ring. Define the \emph{center of $R$} to be
    \begin{equation*}
        Z(R) = \{a \in R : ar = ra \ \text{for all $r \in R$}\}
    \end{equation*}
    Prove that $Z(R)$ is a commutative subring of $R$.
\end{question}

\begin{question}
    An element $a$ is an \emph{idempotent} if $a^2 = a$.
    \begin{itemize}
        \item[(a)] Prove that the only idempotents in an integral domain are 0 and 1.
        \item[(b)] Find a ring with an idempotent that is not equal to 0 nor 1.
        \item[(c)] Let $R$ be a commutative ring with characteristic 2. Prove that the set $S = \{a \in R : a^2 = a\}$ is a subring of $R$.
    \end{itemize}
\end{question}

\begin{question}
    \begin{itemize}
        \item[(a)] Let $R$ be a commutative ring with identity. Show that the set of units in $R$, $U(R)$, is an abelian group under $\times_R$.
        \item[(b)] Let $F$ be a finite field with $n$ elements. Show that $a^{n - 1} = 1$ for all $a \neq 0 \in F$.
    \end{itemize}
\end{question}

\begin{question}
    Let $F$ be a field and let $K$ be a subset of $F$ with at least two elements. Prove that $K$ is a subfield of $F$ if for any $a, b \in K$, $a - b \in K$ and $ab^{-1} \in K$.
\end{question}

\begin{question}
    \begin{itemize}
        \item[(a)] Let $R$ be a commutative ring with prime characteristic $p$. Show that for $r, s \in R$,
        \begin{itemize}
            \item[(i)] $(r + s)^p = r^p + s^p$
            \item[(ii)] $(r + s)^{p^m} = r^{p^m} + s^{p^m}$ for all positive integers $m$.
            \item[(iii)] The Frobenius map $x \mapsto x^p$ is a ring homomorphism from $R$ to $R$.
        \end{itemize}
        \item[(b)] Give an example of a ring of characteristic 4 and elements $r$ and $s$ such that $(r + s)^4 \neq r^4 + s^4$.
    \end{itemize}
\end{question}

\begin{question}
    Let $\phi : R \to S$ be a ring homomorphism. Let $\phi(R) = \{\phi(r) : r \in R\}$. Prove each of the following statements:
    \begin{itemize}
        \item[(a)] $\phi(0_R) = 0_S$
        \item[(b)] $\phi(-b) = -\phi(b)$ for all $b \in R$
        \item[(c)] $\phi(R)$ is a subring of $S$
        \item[(d)] If $R$ is a commutative subring, then $\phi(R)$ is a commutative subring.
        \item[(e)] Suppose $R$ and $S$ are rings with identities. If $\phi$ is onto, then $\phi(1_R) = 1_S$.
        \item[(f)] If $R$ is a field and $\phi(R) \neq \{0_S\}$ then $\phi(R)$ is a field.
    \end{itemize}
\end{question}

\begin{question}
    Consider the ring $S = \left\{\begin{bmatrix} a & b \\ -b & a \end{bmatrix} : a, b \in \R\right\}$ with matrix addition and matrix multiplication. Show that $\phi : \C \to S$ given by
    \begin{equation*}
        \phi(a + ib) = \begin{bmatrix} a & b \\ -b & a \end{bmatrix}
    \end{equation*}
    is a ring isomorphism.
\end{question}

\begin{question}
    Show that $\Z_3[i]$ is ring isomorphic to $\Z_3[x]/\langle{x^2 + 1}\rangle$.
\end{question}

\end{document}
